\chapter{\abstractname}
This thesis addresses the problem of 3D \glsentrylong{mot} for RGB-based systems.
More specifically, we propose a method that performs sparse feature-based object-level \glsentrylong{ba} for accurate object track localization.
Using the 2D object detector and tracker TrackR-CNN \cite{voigtlaenderMOTSMultiObjectTracking2019}, we introduce a procedure for stereo object detections and improve TrackR-CNN's trained association ability.
We achieve superior association via a multi-stage association pipeline that combines appearance and 3D localization similarity.
Additionally, we leverage a priori knowledge of object shapes and dynamics for both association and keeping a sparse outlier-free point cloud representation of objects.

We evaluate our proposal on the KITTI \cite{geigerVisionMeetsRobotics2013} tracking dataset via the traditional CLEAR \cite{bernardinEvaluatingMultipleObject2008} and the recently introduced HOTA \cite{luitenHOTAHigherOrder2021} metrics.
However, as the official KITTI tracking benchmark only includes 2D \glsentrylong{mot} evaluation, and the extended 3D evaluation from \cite{wengBaseline3DMultiObject2019} only supports CLEAR via 3D \glsentrylong{iou}, we implement a customized
tracking ability assessment.
The evaluation introduces a normalized 3D \glsentrylong{giou} \cite{rezatofighiGeneralizedIntersectionUnion2019} detection similarity score to the official HOTA evaluation scripts.\footnote{\url{https://github.com/JonathonLuiten/TrackEval}}
We compare our performance to the LiDAR-based AB3DMOT \cite{wengBaseline3DMultiObject2019} for which 3D tracking results are readily available and demonstrate promising results, especially w.r.t. association and rigid, moving objects.
Furthermore, we show the contribution of various features of our system on an overall performance increase of 17\% for cars and 27\% for pedestrians.
The code for the proposed implementation and performance evaluation is publicly available.\footnote{\url{https://github.com/AnselmC/bamot}}
